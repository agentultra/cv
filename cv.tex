%% start of file `template.tex'.
%% Copyright 2006-2012 Xavier Danaux (xdanaux@gmail.com).
%
% This work may be distributed and/or modified under the
% conditions of the LaTeX Project Public License version 1.3c,
% available at http://www.latex-project.org/lppl/.


\documentclass[11pt,a4paper,sans]{moderncv}   % possible options
                                % include font size ('10pt', '11pt'
                                % and '12pt'), paper size ('a4paper',
                                % 'letterpaper', 'a5paper',
                                % 'legalpaper', 'executivepaper' and
                                % 'landscape') and font family ('sans'
                                % and 'roman')

% moderncv themes
\moderncvstyle{casual}                        % style options are
                                % 'casual' (default), 'classic',
                                % 'oldstyle' and 'banking'
\moderncvcolor{blue}                          % color options 'blue'
                                % (default), 'orange', 'green', 'red',
                                % 'purple', 'grey' and 'black'
%\renewcommand{\familydefault}{\sfdefault}    % to set the default
%font; use '\sfdefault' for the default sans serif font, '\rmdefault'
%for the default roman one, or any tex font name
%\nopagenumbers{}                             % uncomment to suppress
%automatic page numbering for CVs longer than one page

% character encoding
%\usepackage[utf8]{inputenc}                  % if you are not using
%xelatex ou lualatex, replace by the encoding you are using
%\usepackage{CJKutf8}                         % if you need to use CJK
%to typeset your resume in Chinese, Japanese or Korean

% adjust the page margins
\usepackage[scale=0.75]{geometry}
\setlength{\hintscolumnwidth}{3.1cm}           % if you want to change
%the width of the column with the dates
%\setlength{\makecvtitlenamewidth}{10cm}      % for the 'classic'
%style, if you want to force the width allocated to your name and
%avoid line breaks. be careful though, the length is normally
%calculated to avoid any overlap with your personal info; use this at
%your own typographical risks...

% personal data
\firstname{James}
\familyname{King}
%\title{Resumé title}                          % optional, remove /
                                % comment the line if not wanted
%\address{}{}    % optional, remove /
                                % comment the line if not wanted
\mobile{+1~(416)~671~6309}                     % optional, remove /
                                % comment the line if not wanted
%\phone{+2~(345)~678~901}                      % optional, remove /
                                % comment the line if not wanted
%\fax{+3~(456)~789~012}                        % optional, remove /
                                % comment the line if not wanted
\email{james@agentultra.com}                          % optional, remove /
                                % comment the line if not wanted
\homepage{agentultra.com}                   % optional, remove /
                                % comment the line if not wanted
\extrainfo{twitter,github,hn @agentultra}            % optional, remove /
                                % comment the line if not wanted
%\photo[64pt][0.4pt]{picture}                  % optional, remove /
                                % comment the line if not wanted;
                                % '64pt' is the height the picture
                                % must be resized to, 0.4pt is the
                                % thickness of the frame around it
                                % (put it to 0pt for no frame) and
                                % 'picture' is the name of the picture
                                % file
\quote{Polymath Provocateur}                            % optional, remove /
                                % comment the line if not wanted

% to show numerical labels in the bibliography (default is to show no
% labels); only useful if you make citations in your resume
%\makeatletter
%\renewcommand*{\bibliographyitemlabel}{\@biblabel{\arabic{enumiv}}}
%\makeatother

% bibliography with mutiple entries
%\usepackage{multibib}
%\newcites{book,misc}{{Books},{Others}}
%----------------------------------------------------------------------------------
%            content
%----------------------------------------------------------------------------------
\begin{document}
%\begin{CJK*}{UTF8}{gbsn}                     % to typeset your resume
%in Chinese using CJK
%-----       resume
%---------------------------------------------------------
\makecvtitle

%% \section{Education}
%% \cventry{year--year}{Degree}{Institution}{City}{\textit{Grade}}{Description}
%% % arguments 3 to 6 can be left empty
%% \cventry{year--year}{Degree}{Institution}{City}{\textit{Grade}}{Description}

\section{Interests and Research}
\cvitem{}{\emph{Formal Methods}} \cvitem{}{I am interested in
  modelling software systems using specification languages such as
  TLA+ in order to better understand systems I design and to check
  their correctness. I am still learning and looking for opportunities
  to use it in my every day work.}
\cvitem{}{\emph{Programming Languages}} \cvitem{}{I am an active
  contributor to the Hy language and gave a presentation on it to
  PyconCA in 2013. I've undertaken research in distributed
  programming languages by implementing the BloomL language in Python
  which I gave a presentation on at PyconCA 2012.}
\cvitem{}{\emph{Cloud Computing}} \cvitem{}{I have been a contributor
  to the OpenStack Cinder and Astara projects and have experience deploying
  OpenStack as well as developing monitoring and administration tools
  around it.}

\section{Languages}
\cvitemwithcomment{Python}{Expert}{Advanced knowledge of AST,
  contributed to development of a Lisp-like front-end language and
  alternative interpreters. Technical reviewer on \emph{Distributed
  Programming in Python} by Pakt Publishing due out in 2016.}
\cvitemwithcomment{C}{Advanced}{Familiar with C89, C99, and modern tools and
  practices. Capable of contributing kernel modules and user-space
  applications.}
\cvitemwithcomment{Common Lisp}{Intermediate}{Versed in more advanced
  features like CLOS and MOP. Can dissassemble code and debug
  implementation issues. Could write eval.}
\cvitemwithcomment{C++}{Adept}{Familiarity with STL, common
  design patterns and idioms. Minor contributions to sizeable C++
  codebases such as Firefox.}

\section{Tools}
\cvdoubleitem{RDBMS}{PostgreSQL, MySQL}{Web Servers}{Apache, Ngninx, Lighttpd}
\cvdoubleitem{NoSQL}{Redis, Riak}{Web}{Django, Flask, Pylons, Pyramid}
\cvdoubleitem{Networking}{Twisted, Tornado, eventlet}{Cloud}{Cinder, Nova,
  Tempest, Swift, Neutron, Astara}
\cvdoubleitem{Platforms}{Intel x86, x86\_64}{OS}{OSX 10.7+, Windows 7,
  Linux 2.6+}
\section{Talks and Presentations}
\subsection{Distributed Programming in Python: A Model for Strong Eventual
Consistency}
\cvlistitem{\link{http://bit.ly/TGumDX}}
\subsection{Hy: A Lisp That Compiles To Python}
\cvlistitem{\link{http://bit.ly/18ipgc3}}
\subsection{Castle Anthrax: Dungeon Generation Techniques}
\cvlistitem{\link{http://bit.ly/1kvPjNT}}
\subsection{Scaling Python}
\cvlistitem{\link{https://www.youtube.com/watch?v=oE2YxsHPIfA}}

\section{Experience}
\subsection{Vocational}
\cventry{2014-Present}{Cloud Consultant}{Dreamhost}{Remote}{}{Various duties:
\begin{itemize}%
\item Maintenance, development, and packaging of core Openstack components
\item Design and develop key features such as utility billing and metering
\end{itemize}}
\cventry{2013-2014}{Contract Developer}{SwiftStack}{Remote}{}{Various duties:
\begin{itemize}%
\item Maintenance, development, and packaging of core Swift web controller
\item Maintenance, development, and automated deployment of swiftstack.com
\end{itemize}}
\cventry{2012}{Senior Developer}{Internap Network
  Services}{Remote}{}{Software developer on the cloud development team
  specializing in Cinder block storage.\newline{}%
Detailed achievements:%
\begin{itemize}%
\item Integrated Cinder into production
%% \item Achievement 2, with sub-achievements:
%%   \begin{itemize}%
%%   \item Sub-achievement (a);
%%   \item Sub-achievement (b), with sub-sub-achievements (don't do
%%     this!);
%%     \begin{itemize}
%%     \item Sub-sub-achievement i;
%%     \item Sub-sub-achievement ii;
%%     \item Sub-sub-achievement iii;
%%     \end{itemize}
%%   \item Sub-achievement (c);
%%   \end{itemize}
%% \item Achievement 3.
\end{itemize}}
\cventry{2011-2012}{Senior Developer}{Polar
  Mobile}{Toronto}{}{Software developer on the API team.\newline{}%
Detailed achievements:%
\begin{itemize}%
\item Developed a test-driven culture
\item Provided mentorship to interns and junior team members
\end{itemize}}
\cventry{2010-2011}{Senior Developer}{Panometrics Inc}{Toronto}{}
{Managed a legacy development project.\newline{}%
Detailed achievements:%
\begin{itemize}%
\item Increased application uptime
\item Replaced several legacy code paths
\end{itemize}}
\cventry{2008-2010}{Software Developer}{Digisphere Inc}{Toronto}{}
{Distributed application development and web application development.\newline{}%
Detailed achievements:%
\begin{itemize}%
\item Designed task processing system that processed large images across
  n-nodes
\item Scaled web application up to millions of unique impressions
  every month
\item Implemented search engine solution that crawled through >5M
  images
\end{itemize}}
\cventry{2004-2007}{Software Developer}{Misc}{Toronto}{}
{Further history available upon request.}

\section{Interests}
\cvitem{Board games}{I'm a huge gaming nerd and I have a collection of
  30+ board games and table top RPGs.}
\cvitem{Game Design}{I've developed my own table top RPG supplement
  and develop video games and demos using a variety of languages and
  tools}
\cvitem{Rock Climbing}{Multi-pitch sport and bouldering}
\cvitem{Running}{I've ran the Warrior Dash 2011 and Spartan Sprint 2012}

%% \renewcommand{\listitemsymbol}{-~}            % change the symbol for
%%                                 % lists

%% \section{Extra 2}
%% \cvlistdoubleitem{Item 1}{Item 4}
%% \cvlistdoubleitem{Item 2}{Item 5\cite{book1}}
%% \cvlistdoubleitem{Item 3}{}

% Publications from a BibTeX file without multibib
%  for numerical labels:
%  \renewcommand{\bibliographyitemlabel}{\@biblabel{\arabic{enumiv}}}
%  to redefine the heading string (``Publications''):
%  \renewcommand{\refname}{Articles}
%\nocite{*}
%\bibliographystyle{plain}
%\bibliography{publications}                   % 'publications' is the
                                % name of a BibTeX file

% Publications from a BibTeX file using the multibib package
%\section{Publications}
%\nocitebook{book1,book2}
%\bibliographystylebook{plain}
%\bibliographybook{publications}              % 'publications' is the
%name of a BibTeX file
%\nocitemisc{misc1,misc2,misc3}
%\bibliographystylemisc{plain}
%\bibliographymisc{publications}              % 'publications' is the
%name of a BibTeX file

\clearpage
%-----       letter
%---------------------------------------------------------
% recipient data
%% \recipient{Company Recruitment team}{Company, Inc.\\123
%%   somestreet\\some city}
%% \date{January 01, 1984}
%% \opening{Dear Sir or Madam,}
%% \closing{Yours faithfully,}
%% \enclosure[Attached]{curriculum vit\ae{}}     % use an optional
%%                                 % argument to use a string other than
%%                                 % ``Enclosure'', or redefine \enclname
%% \makelettertitle

%% Lorem ipsum dolor sit amet, consectetur adipiscing elit. Duis
%% ullamcorper neque sit amet lectus facilisis sed luctus nisl
%% iaculis. Vivamus at neque arcu, sed tempor quam. Curabitur pharetra
%% tincidunt tincidunt. Morbi volutpat feugiat mauris, quis tempor neque
%% vehicula volutpat. Duis tristique justo vel massa fermentum
%% accumsan. Mauris ante elit, feugiat vestibulum tempor eget, eleifend
%% ac ipsum. Donec scelerisque lobortis ipsum eu vestibulum. Pellentesque
%% vel massa at felis accumsan rhoncus.

%% Suspendisse commodo, massa eu congue tincidunt, elit mauris
%% pellentesque orci, cursus tempor odio nisl euismod augue. Aliquam
%% adipiscing nibh ut odio sodales et pulvinar tortor laoreet. Mauris a
%% accumsan ligula. Class aptent taciti sociosqu ad litora torquent per
%% conubia nostra, per inceptos himenaeos. Suspendisse vulputate sem
%% vehicula ipsum varius nec tempus dui dapibus. Phasellus et est urna,
%% ut auctor erat. Sed tincidunt odio id odio aliquam mattis. Donec
%% sapien nulla, feugiat eget adipiscing sit amet, lacinia ut
%% dolor. Phasellus tincidunt, leo a fringilla consectetur, felis diam
%% aliquam urna, vitae aliquet lectus orci nec velit. Vivamus dapibus
%% varius blandit.

%% Duis sit amet magna ante, at sodales diam. Aenean consectetur porta
%% risus et sagittis. Ut interdum, enim varius pellentesque tincidunt,
%% magna libero sodales tortor, ut fermentum nunc metus a ante. Vivamus
%% odio leo, tincidunt eu luctus ut, sollicitudin sit amet metus. Nunc
%% sed orci lectus. Ut sodales magna sed velit volutpat sit amet pulvinar
%% diam venenatis.

%% Albert Einstein discovered that $e=mc^2$ in 1905.

%% \[ e=\lim_{n \to \infty} \left(1+\frac{1}{n}\right)^n \]

%% \makeletterclosing

%\clearpage\end{CJK*}                         % if you are typesetting
%your resume in Chinese using CJK; the \clearpage is required for
%fancyhdr to work correctly with CJK, though it kills the page
%numbering by making \lastpage undefined
\end{document}


%% end of file `template.tex'.
